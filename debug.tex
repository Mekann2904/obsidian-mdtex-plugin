% Options for packages loaded elsewhere
\PassOptionsToPackage{unicode}{hyperref}
\PassOptionsToPackage{hyphens}{url}
\documentclass[
  8pt,
]{extarticle}
\usepackage{xcolor}
\usepackage[margin=25mm]{geometry}
\usepackage{amsmath,amssymb}
\setcounter{secnumdepth}{-\maxdimen} % remove section numbering
\usepackage{iftex}
\ifPDFTeX
  \usepackage[T1]{fontenc}
  \usepackage[utf8]{inputenc}
  \usepackage{textcomp} % provide euro and other symbols
\else % if luatex or xetex
  \usepackage{unicode-math} % this also loads fontspec
  \defaultfontfeatures{Scale=MatchLowercase}
  \defaultfontfeatures[\rmfamily]{Ligatures=TeX,Scale=1}
\fi
\usepackage{lmodern}
\ifPDFTeX\else
  % xetex/luatex font selection
\fi
% Use upquote if available, for straight quotes in verbatim environments
\IfFileExists{upquote.sty}{\usepackage{upquote}}{}
\IfFileExists{microtype.sty}{% use microtype if available
  \usepackage[]{microtype}
  \UseMicrotypeSet[protrusion]{basicmath} % disable protrusion for tt fonts
}{}
\makeatletter
\@ifundefined{KOMAClassName}{% if non-KOMA class
  \IfFileExists{parskip.sty}{%
    \usepackage{parskip}
  }{% else
    \setlength{\parindent}{0pt}
    \setlength{\parskip}{6pt plus 2pt minus 1pt}}
}{% if KOMA class
  \KOMAoptions{parskip=half}}
\makeatother
\usepackage{longtable,booktabs,array}
\usepackage{calc} % for calculating minipage widths
% Correct order of tables after \paragraph or \subparagraph
\usepackage{etoolbox}
\makeatletter
\patchcmd\longtable{\par}{\if@noskipsec\mbox{}\fi\par}{}{}
\makeatother
% Allow footnotes in longtable head/foot
\IfFileExists{footnotehyper.sty}{\usepackage{footnotehyper}}{\usepackage{footnote}}
\makesavenoteenv{longtable}
\usepackage{graphicx}
\makeatletter
\newsavebox\pandoc@box
\newcommand*\pandocbounded[1]{% scales image to fit in text height/width
  \sbox\pandoc@box{#1}%
  \Gscale@div\@tempa{\textheight}{\dimexpr\ht\pandoc@box+\dp\pandoc@box\relax}%
  \Gscale@div\@tempb{\linewidth}{\wd\pandoc@box}%
  \ifdim\@tempb\p@<\@tempa\p@\let\@tempa\@tempb\fi% select the smaller of both
  \ifdim\@tempa\p@<\p@\scalebox{\@tempa}{\usebox\pandoc@box}%
  \else\usebox{\pandoc@box}%
  \fi%
}
% Set default figure placement to htbp
\def\fps@figure{htbp}
\makeatother
\setlength{\emergencystretch}{3em} % prevent overfull lines
\providecommand{\tightlist}{%
  \setlength{\itemsep}{0pt}\setlength{\parskip}{0pt}}
\renewcommand{\lstlistingname}{code}
\renewcommand{\lstlistlistingname}{code}
\usepackage{bookmark}
\IfFileExists{xurl.sty}{\usepackage{xurl}}{} % add URL line breaks if available
\urlstyle{same}
\hypersetup{
  hidelinks,
  pdfcreator={LaTeX via pandoc}}

\author{}
\date{}

\begin{document}

\begin{lstlisting}[language=Python]
print(x)

\end{lstlisting}

\begin{lstlisting}[label={lst:aaa},caption={aaa}]
print(x)

\end{lstlisting}

\begin{lstlisting}[language=Python,label={lst:bbb},caption={bbb}]
print(x)

\end{lstlisting}

\begin{lstlisting}[language=Python]
import numpy as np
from dataset.mnist import load\_mnist
from two\_layer\_net import TwoLayerNet

(x\_train, t\_train), (x\_test, t\_test) = load\_mnist(normalize=True)

train\_loss\_list = []
train\_acc\_list = []
test\_acc\_list = []

\# ハイパーパラメータ
iters\_num = 10000
batch\_size = 100
learning\_rate = 0.1

\# 1エポックあたりの繰り返し数
train\_size = x\_train.shape[0]
iter\_per\_epoch = max(train\_size / batch\_size, 1) \# 訓練データの総数 / ミニバッチサイズ

network = TwoLayerNet(input\_size=784, hidden\_size=50, output\_size=10)

for i in range(iters\_num):
    \# ミニバッチの取得
    batch\_mask = np.random.choice(train\_size, batch\_size)
    x\_batch = x\_train[batch\_mask]
    t\_batch = t\_train[batch\_mask]

    \# 勾配の計算
    grad = network.numerical\_gradient(x\_batch, t\_batch)
    \# grad = network.gradient(x\_batch, t\_batch) \# 高速版(誤差逆伝播法)を使用することも可能

    \# パラメータの更新
    for key in ('W1', 'b1', 'W2', 'b2'):
        network.params[key] -= learning\_rate * grad[key]

    loss = network.loss(x\_batch, t\_batch)
    train\_loss\_list.append(loss)

    \# 1エポックごとに認識精度を計算
    if i \% iter\_per\_epoch == 0:
        train\_acc = network.accuracy(x\_train, t\_train)
        test\_acc = network.accuracy(x\_test, t\_test)
        train\_acc\_list.append(train\_acc)
        test\_acc\_list.append(test\_acc)
        print("train acc, test acc | " + str(train\_acc) + ", " + str(test\_acc))

\end{lstlisting}

コード

\begin{figure}
\centering
\pandocbounded{\includegraphics[keepaspectratio,alt={aaa}]{/Users/mekann/plugin-dev/Pasted image 20251201191155.png}}
\caption{aaa}\label{fig:aaa}
\end{figure}

\begin{figure}
\centering
\includegraphics[width=100mm,height=\textheight,keepaspectratio,alt={bbb}]{/Users/mekann/plugin-dev/Pasted image 20251201191210.png}
\caption{bbb}\label{fig:bbb}
\end{figure}

\end{document}
